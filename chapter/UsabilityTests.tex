\section{Usability Tests}
\subsection{Tagebuchstudie}
\subsection{Card Sorting}
Das Card Sorting ist eine Möglichkeit die Struktur und Menüführung einer Seite zu testen.
Es werden Karten erstellt, die die wichtigsten Inhalte umschreiben.
Die Tester schreiben dann den für sie passenden Navigationsbegriff auf die Karte.
Die so entstandenen Karten werden dann von den Testern nach Inhalt gruppiert und diese Gruppen mit Oberbegriffen versehen.
Eine Person dokumentiert, wobei die Tester sich einig waren und welche Punkte umstritten waren.
Dies kann sowohl in Person als auch online erfolgen.\\\\
Diese Testmethode sollte angewendet werden, wenn die Strukturierung der Seite und des Menüs überarbeitet werden soll.
Sie braucht kleinere Gruppen von Testern hat aber einen mittleren Aufwand, da bei jeder Gruppe ein Moderator, der das Vorgehen dokumentiert, erforderlich ist.\autocite[vgl.~][]{usability.de.cardsorting}

\subsection{Eye-Tracking}\label{subsec:eyetracking}
Eyetracking wird meist in Kombination mit anderen Usability Tests durchgeführt wie Labortests oder ähnliches.
Beim Eyetracking wird aufgezeichnet auf welche Bereiche des Bildschirms, also der Seite ein Nutzer schaut und wie lange.
Dadurch entstehen neben den subjektiven Daten, die der Nutzer selbst berichtet weitere objektive Daten.
Diese Daten helfen bei der Auswertung der Tests massiv.\\\\
Diese Testmethode sollte angewendet werden, wenn herauszufinden ist, wo wichtige Elemente der Seite platziert werden sollen, wie zum Beispiel ein Kauf-Button,
oder welche Elemente entfernt werden können, da diese von Nutzern nicht beachtet werden.\autocite[vgl.~][]{usability.de.eyetracking}

\subsection{Onsite Befragung}
Onsite Befragungen werden auf der Seite durchgeführt, somit werden nur wirkliche Nutzer der Seite befragt.
Diese sind meist in der Form von Befragungen umgesetzt, die als Pop-up-Fenster angezeigt werden.
Diese Pop-Ups können entweder direkt beim Besuch der Seite angezeigt werden oder nach dem durchführen bestimmter Interaktionen.
So kann zum Beispiel nach dem Bestellprozess eine Befragung angezeigt werden, wie zufrieden der Kunde damit war.\autocite[vgl.~][]{usability.de.methods}\\\\
Diese Testmethode erreicht mit etwas anfänglichem Aufwand viele Nutzer und kann sogar automatisch ausgewertet werden.
Somit eignet sie sich sehr gut für Befragungen mit einfachen Auswahlmöglichkeiten.\autocite[vgl.~][]{usability.de.onsite}

\subsection{Usability Test im Labor}
Usability Tests im Labor geben Nutzern eine spezielle Aufgabe, die sie ausführen müssen.
Dies kann das Finden bestimmter Ergebnisse auf einer Seite oder aber der Einkauf eines bestimmten Produkts sein.
Die Tester werden dabei von Experten überwacht und äußern \("\)[\ldots]ihre Empfindungen und Gedanken zur Nutzerfreundlichkeit der Seite\("\)\autocite{mso.usability}.\\\\
Diese Testmethode sollte für spezielle wichtige Szenarien angewendet werden, da sie sehr aufwändig ist.
Tester und Experten müssen vor Ort sein und Ausrüstungen und Materialien werden benötigt.
Es bietet sich für noch bessere Ergebnisse an diese Testmethode mit weiteren wie z.B.~\ref{subsec:eyetracking} zu verbinden.