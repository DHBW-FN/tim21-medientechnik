%! Author = philippkust
%! Date = 17.07.22


% Document
\begin{document}

\section{Konkurrenzanalyse}
Diese Arbeit beginnt mit einer Konkurrenzanalyse der Webseiten www.google.com, www.bing.com sowie www.qwant.com.
Dabei wird insbesondere auf den Inhalt der jeweiligen Webseiten, die Zielgruppenorientierung, das Design, die Informations-
und Navigationsarchitektur, die Texte sowie auf die Technik der Seiten eingegangen. Zusätzlich zu diesen Punkten wird auch
auf die Qualität der Suchvorschläge eingegangen.

\subsection{Inhalt}
Das Ziel aller drei Webseiten ist es, den Nutzern die Möglichkeit zu bieten, das Internet nach Webseiten zu einem gewissen
Thema zu durchsuchen. Ohne das Vorwissen, dass die drei Seiten dies anbieten, ist diese Dienstleistung nicht auf den ersten
Blick erkenntlich.
\begin{figure}[h]
    \centering
    \includegraphics{Google Startseite}
    \caption{Google Startseite}
\end{figure}

Beispielsweise besteht die Startseite von Google nur aus einer Leiste, in der man etwas eingeben kann.
Ohne eine Anleitung, was sie in dieser Leiste eingeben können, führt dies am Anfang für Verwirrung. Nutzer, die mit Google
vertraut sind, profitieren allerdings von dieser Anordnung, da diese nicht von der Funktionalität ablenkt.
\begin{figure}[h]
    \centering
    \includegraphics{Bing Startseite}
    \caption{Bing Startseite}
\end{figure}
Einen ähnlichen
Aufbau bietet Microsoft mit ihrer Suchmaschine Bing. Hierbei werden standardmäßig allerdings noch aktuelle Nachrichten sowie
ein Hintergrundbild angezeigt.
\begin{figure}[h]
    \centering
    \includegraphics{Qwant Startseite}
    \caption{Qwant Startseite}
\end{figure}
Einzig Qwant folgt diesem Schema nicht ganz. Bei dieser Seite ist die Suchleiste, welche die
Hauptfunktion ist, im oberen Bereich, wodurch der Fokus auch auf die Texte in der Mitte des Bildschirms fallen kann. Dies
erschwert es Erstanwendern noch mehr als bei den Konkurrenten, die Funktion intuitiv zu bedienen. Somit mangelt es allen
Suchmaschinen an Selbstbeschreibungsfähigkeit. Dies wäre gewährleistet, „wenn jeder einzelne Dialogschritt unmittelbar
verständlich ist“\cite[Seite 6]{Maulhardt.20220506}.
. Auch adäquate Anleitungen werden nicht zur Verfügung gestellt

Trotz der oben genannten Kritik bieten alle drei Suchmaschinen, insbesondere für erfahrene Anwender, einen echten Mehrwert.
So erleichtern es alle Seiten, sich im Internet zurechtzufinden und ermöglichen das Finden bestimmter Webseiten.

\subsection{Zielgruppenorientierung}
Wie jedes Produkt haben auch diese drei Webseiten Zielgruppen, welche durch verschiedene Aspekte wie Design oder Inhalt
angesprochen werden sollen. Diese Zielgruppen sollen nachfolgend herausgearbeitet werden und die Orientierung an diesen
bewertet werden.

Um herauszufinden, wie gut die drei Seiten ihre Zielgruppen abbilden, ist es zuerst nötig, diese Zielgruppen zu
identifizieren. Diese Adressaten haben bei allen drei Seiten eine große Schnittmenge, unterscheiden sich allerdings in
Kleinigkeiten. Da alle drei Unternehmen, welche die Suchmaschinen betreiben, einen möglichst großen Umsatz durch Werbeplatzierungen
erzielen wollen, versuchen diese, eine möglichst große Zahl an Usern zu generieren, weswegen die Seiten auch auf diesen Zweck
zugeschnitten sind.  Da nicht alle Menschen Zugang zu Internet haben, richten sich die Webseiten insbesondere an Personen
mit diesem Zugang. Ansonsten ist die Bandbreite an potenziellen Nutzergruppen sehr groß. Dies erkennt man auch an der großen
Auswahl an Sprachen, in welchen die Webseiten dargestellt werden können. So bietet Google knapp 150 Sprachen an, Bing
knapp 100. Qwant müsste sich hingegen noch breiter aufstellen. Mit gerade einmal 12 Sprachen wird eine große Zahl an
potenziellen Nutzern nicht angesprochen.

Eine Besonderheit gibt es bei der Zielgruppe von Qwant jedoch noch zu beachten. Sie versuchen insbesondere Leute anzusprechen,
denen Privatsphäre wichtig ist. Dies wird bereits auf der Startseite klar. Auf dieser wird damit geworben, dass die Suchmaschine
nichts über die Nutzer weiß, da diese keine persönlichen Daten erheben oder die Websuche tracken (siehe Abbildung \ref{fig:qwantstartseite}).
\begin{figure}[h]
    \centering
    \includegraphics{Qwant Startseite}
    \caption{Qwant Startseite}
    \label{fig:qwantstartseite}
\end{figure}

Ebenfalls spricht die Suchmaschine Qwant eine kindlichere Zielgruppe bzw. deren Eltern an. In einem klar abgegrenzten
Bereich wird es den Eltern ermöglicht, für die Kinder eine sichere Suche im Internet zu gewährleisten. Bei den Konkurrenten
fehlt ein solches Angebot.

\subsection{Design}
Ein wichtiger Punkt einer Webseite ist dessen Design. „Unser visuelles System ist darauf ausgelegt, schnell zu entscheiden,
ob wir etwas gut oder schlecht finden. Dabei wird der Ersteindruck einer Website vor allem durch die Schönheit, also die
visuelle Ästhetik, bestimmt.“\cite[Seite 43]{Thielsch.}.
Somit ist das Design eines
der zentralen Elemente, welches besonders gut ausgearbeitet sein muss. Dabei gehen alle drei Suchmaschinen einen anderen Weg.

Googles Auftritt ist sehr minimalistisch gehalten. Bis auf ein Logo, welches sich in unregelmäßigen Zeitabständen verändert,
ist nur die Suchleiste in der Bildmitte zu sehen. Der Hintergrund ist dabei je nach Einstellung des Dark-Modes dunkelgrau
oder weiß. Diese fehlende Farbauswahl spiegelt nicht direkt das Firmenimage von Google wider, welches, wie am Farbenfrohen
Firmenlogo zu erkennen, eher verspielt, modern und innovativ wirkt.
\begin{figure}[h]
    \centering
    \includegraphics{Google Firmenlogo}
    \caption{Firmenlogo Google\cite{.2020}}
\end{figure}

Im Gegensatz zu diesem einfarbigen Design von Google bietet Bing die Möglichkeit ein Landschaftsbild als Hintergrund zu
setzen, welches sich täglich ändert. Diese Veränderung bringt regelmäßige Abwechslung, was auf den Nutzer erfrischend wirken
kann. Außerdem wirken die Bilder freundlich auf die Nutzer, weswegen die Startseite von Bing gegenüber der von Google
bevorzugt werden könnte.
\begin{figure}[h]
    \centering
    \includegraphics{Bing Startseite 2}
    \caption{Bing Startseite 2}
\end{figure}

Am auffälligsten und damit auch am meisten Raum für Analyse bietend ist allerdings die Startseite von Qwant. Mehrere
klassische Formen des Kontrasts wurden hier benutzt. Zum einen ist der starke Kontrast zwischen gelb und blau auffällig.
\begin{figure}[h]
    \centering
    \includegraphics{Qwant Startseite Design}
    \caption{Qwant Startseite Design}
\end{figure}
Dies erinnert stark an einen Komplementärkontrast. Dies bedeutet, dass sich die beiden Farben im Farbenkreis gegenüberliegen\cite[Seite 33]{Maulhardt.20220513}.
Bei Betrachtung des Farbkreises nach Itten (\ref{fig:farbkreis}) fällt dies ebenfalls auf.
\begin{figure}[h]
    \centering
    \includegraphics{Farbkreis}
    \caption{Farbkreis nach Itten\cite{Farbkreis}}
    \label{fig:farbkreis}
\end{figure}
Auch ein
Warm-Kalt-Kontrast ist vorhanden. Hierbei wird die warme Farbe Gelb der kalten Farbe Blau direkt gegenübergestellt\cite[Seite 34]{Maulhardt.20220513}.
Zuletzt beinhaltet die Startseite noch einen Farbe-an-sich-Kontrast.
\begin{figure}[h]
    \centering
    \includegraphics{Qwant Startseite 3}
    \caption{Startseite von Qwant}
    \label{fig:qwantstartseite3}
\end{figure}
Dieser ergibt sich dadurch,
dass wie bei \ref{fig:qwantstartseite3} die Grundfarben Blau und Lila gegenübergestellt werden\cite[Seite 38]{Maulhardt.20220513}. All diese Kontraste
lassen die Startseite von Qwant lebendig wirken, was mit dem Image der Firma bzw. der Suchmaschine übereinstimmt.
Trotzdem wäre es wünschenswert, wenn das Erscheinungsbild angepasst werden könnte, da dies einigen aufgrund der starken
Kontraste abschrecken könnte. Diese junge Zielgruppe ist auch aufgrund der Benutzung von Emojis in beispielsweise den
Filtereinstellungen erkennbar (\ref{fig:qwantfilter}).
\begin{figure}[h]
    \centering
    \includegraphics{Qwant Filter}
    \caption{Suchfiltereinstellungen von Qwant}
    \label{fig:qwantfilter}
\end{figure}

Positiv fällt bei allen drei Seiten auf, dass diese einen einstellbaren Dark Mode besitzen, welcher es ermöglicht, das
Design zu verändern. Zusätzlich bietet Bing die Möglichkeit das oben angesprochene Hintergrundbild nicht anzeigen zu lassen,
sodass die Optik mit entweder dunkelgrau oder weiß als Hintergrund sehr stark Google ähnelt.

\subsection{Informations- und Navigationsarchitektur}
Die Navigationsarchitektur ist auf allen drei Seiten relativ gleich aufgebaut. Die Startseite besteht dabei aus einer
Suchleiste, welche nach der Suche an den oberen Bildrand geht, wobei die vorgeschlagenen Webseiten darunter angezeigt werden.
Das Logo der Suchmaschine dient dabei immer als Homebutton. Diese Grundstruktur ist für Personen, die die Funktionalität der
Seiten kennen sehr einfach zu durchschauen, da sie sich von Anbieter zu Anbieter nicht verändert. Allerdings unterscheidet
sich die Architektur in Kleinigkeiten. Darauf wird im Nachfolgenden eingegangen.

Google bietet in der rechten oberen Ecke die Möglichkeit, den Account zu managen, auf zahlreiche andere Google Dienste zu
wechseln oder eine Bilder\-Suche durchzuführen. Am unteren Bildrand wird zum einen auf Informationsseiten zu Google verwiesen,
wie auch die Möglichkeiten geboten, die Einstellungen zu ändern.
\begin{figure}[h]
    \centering
    \includegraphics{Google Einstellungen}
    \caption{Google Einstellungen}
\end{figure}
Diese Auswahlmöglichkeiten sind allerdings sehr klein gehalten,
wodurch sie für Leute, welche nicht mehr so gut sehen können, nicht mehr gut erkennbar sind. Auch ein geringer Kontrast
erschwert die Lesbarkeit dieser Elemente.

Bing hat im Gegensatz zu Google ein paar Punkte, welche positiv aufgefallen sind. Insbesondere die Möglichkeit, sich Newsartikel
sich anzeigen zu lassen, auf welche man bei Bedarf klicken kann, stach dabei im direkten Vergleich heraus. Ein weiterer Punkt
ist die Verwendung von Icons, welche bei Google nur sehr wenig eingesetzt werden. Dahingegen schafft es Bing, diese gezielt
und unterstützend zu normalem Test einzusetzen, was die Bedienung vereinfacht.
\begin{figure}[h]
    \centering
    \includegraphics{Bing Icons 1}
    \caption{Google Icons}
\end{figure}
Diese Icons sind dabei auch gemäß den Regeln
für die Gestaltung einer Navigationsleiste\cite[Seite 17]{Maulhardt.20220621b} selbsterklärend, da sie alle standardmäßig auch von
anderen Webseiten für dieselben Funktionen verwendet werden.

Zuletzt wird die Navigationsstruktur von Qwant analysiert. Diese hat sowohl Vor- als auch Nachteile gegenüber den Konkurrenten.
Zum einen ist die Zahl an Auswahlpunkten, zumindest wenn auf der Startseite nicht gescrollt wird, sehr leicht überschaubar.
So kann man nur zwischen Maps, den Produkten, Informationen über Qwant und den Nachrichten des Tages wählen. Diese geringe
Zahl ermöglicht es, vor allem im Vergleich zu den Konkurrenten, welche 13 (Google) und 9 (Bing) Auswahlpunkte direkt auf der
Startseite haben, schnell einen Überblick über die Struktur der Seite zu erlangen. Damit halten sich die beiden Seiten auch
nicht an die Regel, dass nur maximal sieben Buttons gleichzeitig zu sehen sind, da das Gehirn nicht mehr auf einen Blick
wahrnehmen kann\cite[Seite 16]{Maulhardt.20220621b}. Negativ wiederum an der Startseite von Qwant ist, dass der Punkt „Maps“, welcher
in Abbildung \ref{fig:qwantnavigation} zu sehen ist, sich mit dem Menü, welches unter Produkte gezeigt wird, überschneidet.
\begin{figure}[h]
    \centering
    \includegraphics{Qwant Navigation}
    \caption{Qwant Navigation}
    \label{fig:qwantnavigation}
\end{figure}

So werden dabei die
Produkte Search und Maps vorgeschlagen, obwohl man auf Search bereits ist und Maps einfacher über den extra Butten Maps
aufrufbar ist. Einzig das Produkt „Junior“ ist in dieser Reihe neu. Ebenso negativ anzumerken ist, dass die Einstellungen
schwer zu finden sind. So muss der Nutzer bis ganz nach unten scrollen, damit sehr klein unten rechts der Button Einstellungen
zu sehen ist. Dieser ist zusätzlich nicht einmal deutlich hervorgehoben, wie in Abbildung \ref{fig:qwanteinstellungen} zu sehen ist.
\begin{figure}[h]
    \centering
    \includegraphics{Qwant Einstellungen}
    \caption{Qwant Einstellungen}
    \label{fig:qwanteinstellungen}
\end{figure}

\subsection{Text}
Aufgrund der Ziele der Webseitenbetreiber sind die Seiten mit wenig Text versehen. Stattdessen ist der Aufbau der Seiten
so gestaltet, dass der Nutzer sich intuitiv zurechtfindet. Außerdem ist es nicht das Ziel der Suchmaschinen den User durch
lange Texte vom eigentlichen Sinn der Seite abzulenken, nämlich dem Suchen nach anderen Webseiten. Einzig Qwant fällt aus
diesem Muster heraus. Sie versuchen den Besucher auf der Homepage von ihrem Konzept der privaten Suche mit der Hilfe von
Text zu überzeugen. Dabei befolgen Sie auch alle Regeln der Textgestaltung\cite[Seite 5ff]{Maulhardt.20220621b}. Durch kurze prägnante Texte,
in welchen durch fette Schrift das wichtigste markiert ist, wird die Kernaussage gut dargestellt.
\begin{figure}[h]
    \centering
    \includegraphics{Qwant Texte}
    \caption{Qwant Texte}
\end{figure}

\subsection{Technik}
Ebenso wichtig wie die bisher angesprochenen Analysekriterien ist die Technik, mit welcher die Webseiten gestaltet wurden.
So können beispielsweise lange Ladezeiten dafür sorgen, dass der Nutzer eine andere Suchmaschine in Zukunft benutzen wird.
Um diese Technik möglichst objektiv bewerten zu können wurde ein standardisierter Test durchgeführt, welcher Webseiten nach
vier Analysekriterien beurteilt. Dieser Test namens „Lighthouse“ ist dabei eine Chrome-Erweiterung.

Google schneidet dabei am besten ab. Nur im Bereich SEO, was für Search Engine Optimization steht, ist noch wirkliches
Potenzial, die Seite zu verbessern. Ansonsten ist Google in den drei Bereichen Performance Accessibility und Best Practices
in einem sehr guten Bereich. Insbesondere die Performance, welche bei den anderen beiden Anbietern schlechter ist,
sticht dabei heraus.
\begin{figure}[h]
    \centering
    \includegraphics{Google Technik}
    \caption{LightHouse Analyseergebnisse für Google}
\end{figure}

Bing hingegen hat insbesondere im Bereich Performance noch Verbesserungsbedarf. Hierbei wird empfohlen, ungenutzten JavaScript
Code zu reduzieren. Ebenso könnte der Analysepunkt Accessibility besser sein.
\begin{figure}[h]
    \centering
    \includegraphics{Bing Technik}
    \caption{LightHouse Analyseergebnisse für Bing}
\end{figure}

Qwant schneidet im Bereich Performance mit gerade einmal 54 von 100 möglichen Punkten am schlechtesten ab. Dabei wird ebenfalls
geraten, ungenutzten JavaScript Code zu reduzieren.
\begin{figure}[h]
    \centering
    \includegraphics{Qwant Technik}
    \caption{LightHouse Analyseergebnisse für Qwant}
\end{figure}

\subsection{Qualität der Suchvorschläge}
Bei der Analyse einer Suchmaschine sind auch die Suchvorschläge sowie die Ergebnisse zu evaluieren. Dies wurde exemplarisch
an dem Begriff Medientechnik untersucht. Google und Bing sind dabei auf einem vergleichbaren Niveau. Beide Suchmaschinen
bieten zahlreiche Möglichkeiten zum bisher eingegebenen Begriff „Mediente“, wie aus den Abbildungen \ref{fig:googlevervollstandigung} und \ref{fig:bingvervollstandigung} hervorgeht.
\begin{figure}[h]
    \centering
    \includegraphics{Google Vervollständigung}
    \caption{Google Vervollständigung für den Begriff "Mediente"}
    \label{fig:googlevervollstandigung}
\end{figure}

\begin{figure}[h]
    \centering
    \includegraphics{Bing Vervollständigung}
    \caption{Bing Vervollständigung für den Begriff "Mediente"}
    \label{fig:bingvervollstandigung}
\end{figure}

Ebenso werden zahlreiche verschiedene Informationen nach der Suche präsentiert, welche über die reine Auflistung von Webseiten
hinausgehen. So boten beide Webseiten Unternehmen in der Nähe des Benutzers an. Bing präsentierte zusätzlich Videos, wohingegen
Google auf den Wikipedia Artikel zu Medientechnik verwies.

Dahingegen bot Qwant keine Vervollständigung des Begriffs „Mediente“ an. Ebenso besteht Verbesserungsbedarf in der Präsentation
der Webseiten. In diesem Fall wurden nur mögliche Webseiten aufgelistet (siehe Abbildung \ref{fig:qwantvorschlage}), während beispielsweise Google
interaktive Schalflächen zu Maps und Wikipedia anbot, wie aus Abbildung \ref{fig:googlevorschlage} hervorgeht.
\begin{figure}[h]
    \centering
    \includegraphics{Qwant Suchvorschläge}
    \caption{Ergebnis der Suche nach "Medientechnik" bei Qwant}
    \label{fig:qwantvorschlage}
\end{figure}

\begin{figure}[h]
    \centering
    \includegraphics{Google Suchvorschläge}
    \caption{Ergebnis der Suche nach "Medientechnik" bei Google}
    \label{fig:googlevorschlage}
\end{figure}

\end{document}